\chapter{Implementierung \& Test}
\label{chap: Implementierung und Test}
In diesem Kapitel...

% Reihenfolge Grammatik & Grammatik Generator??
\section{Gedcom Grammatik}
\label{sec: Implementierung - Gedcom Grammatik}
% zeigen wie eine Grammatik aussieht
% Zeigen wie der Postprocessor funktioniert (auch Ausschnitt von Lexer möglich)
% Problem mit Ambigious Grammar -> erklären was das Problem ist und wie gelöst wurde
\subsection{Gedcom7 Syntax in Nearley}
\label{subsec: Gedcom7 Syntax in Nearley}
Da die Gedcom7- sowie die Nearley Syntax beide auf EBNF-Sprachkonzepten basieren, lässt sich die Gedcom7 Spezifikation ohne weiteres in eine Nearley Grammatik übersetzten. Um Nearley Regeln für eine \textit{Gedcom Line}\footnote{siehe \hyperref[sec: GEDCOM Version 7]{Kapitel GEDCOM Version 7}} zu definieren, können die folgenden Tokens für das Leerzeichen, den \textit{Cross-Reference Identifier} und die End-Of-Line Zeichenfolge in Form von regulären Ausdrücken definiert werden:
\\ \\
\begin{minipage}{1.0\textwidth} \small
	\begin{lstlisting}
		D    : /[ ]/
		Xref : /\@[A-Z0-9\_]+\@/	
		EOL  : /(?:\r\n?|\n)/
	\end{lstlisting}
	\captionof{lstlisting}{Tokens für eine Gedcom Line, definiert als regulärer Ausdruck }
	\label{lst: tokens gedcom line}
\end{minipage}
\\ \\
Diese regulären Ausdrücke werden in der Vorverarbeitungsphase vom Moo-Lexer verwendet, um zusammenhängende Zeichen zu Tokens zu gruppieren, die dann in der Nearley Grammatik über den Tokennamen mit einem vorangestellten \%-Zeichen angesprochen werden können. Soll nun die erste Line eines Individual-Records geparsed werden, könnte dies mit der folgenden Nearley-Regel umgesetzt werden:
\\ \\
\begin{minipage}{1.0\textwidth} \small
	\begin{lstlisting}
		record_Indi -> "0"  %D  %Xref  %D  "INDI"  %EOL 
	\end{lstlisting}
	\captionof{lstlisting}{Nearley Regel zum parsen eines Individual Records}
	\label{lst: nearley regel individual record}
\end{minipage}
\\ \\
Diese Regel würde die erste Line des in \hyperref[lst: individual record example]{Listing XY} vorgestellten Individual-Records als Eingabe akzeptieren.


\section{Grammatik Generator}
\label{sec: Implementierung - Grammatik Generator}
% Klassendiagramm mit Ablauf 


\section{Gedcom Struktur}
\label{sec: Implementierung - Gedcom Struktur}
% Klassendiagramm 


\section{Gedcom Parser}
\label{sec: Implementierung - Gedcom Parser}
% Kurze Beschreibung wie alles zusammengefügt wird
% Beispielhafter Ablauf


