\chapter{Implementierung \& Test}
\label{chap: Implementierung und Test}
In diesem Kapitel wird beschrieben wie das im Kapitel \ref{chap: Konzept} vorgestellte Konzept in der Bibliothek \textit{gedcom7.js} implementiert wird. 

% Reihenfolge Grammatik & Grammatik Generator??
\section{Gedcom Grammatik}
\label{sec: Implementierung - Gedcom Grammatik}
% zeigen wie eine Grammatik aussieht
% Zeigen wie der Postprocessor funktioniert (auch Ausschnitt von Lexer möglich)
% Problem mit Ambigious Grammar -> erklären was das Problem ist und wie gelöst wurde
\subsection{Gedcom7 Syntax in Nearley}
\label{subsec: Gedcom7 Syntax in Nearley}
Da die Gedcom7- sowie die Nearley Syntax beide auf EBNF-Sprachkonzepten basieren, lässt sich die Gedcom7 Spezifikation ohne weiteres in eine Nearley Grammatik übersetzten. Um Nearley Regeln für eine \textit{Gedcom Line}\footnote{siehe \hyperref[sec: GEDCOM Version 7]{Kapitel GEDCOM Version 7}} zu definieren, können die folgenden Tokens für das Leerzeichen, den \textit{Cross-Reference Identifier} und die End-Of-Line Zeichenfolge in Form von regulären Ausdrücken definiert werden:
\\ \\
\begin{minipage}{1.0\textwidth} \small
	\begin{lstlisting}
		D    : /[ ]/
		Xref : /\@[A-Z0-9\_]+\@/	
		EOL  : /(?:\r\n?|\n)/
	\end{lstlisting}
	\captionof{lstlisting}{Tokens für eine Gedcom Line, definiert als regulärer Ausdruck }
	\label{lst: tokens gedcom line}
\end{minipage}
\\ \\
Diese regulären Ausdrücke werden in der Vorverarbeitungsphase vom Moo-Lexer verwendet, um zusammenhängende Zeichen zu Tokens zu gruppieren, die dann in der Nearley Grammatik über den Tokennamen mit einem vorangestellten \%-Zeichen angesprochen werden können. Soll nun die erste Line eines Family-Records geparsed werden, könnte dies mit der folgenden Nearley-Regel umgesetzt werden:
\\ \\
\begin{minipage}{1.0\textwidth} \small
	\begin{lstlisting}
		record_FAM -> "0"  %D  %Xref  %D  "FAM"  %EOL 
	\end{lstlisting}
	\captionof{lstlisting}{Nearley Regel zum parsen eines Family Records}
	\label{lst: nearley regel family record first line}
\end{minipage}
\\ \\
Diese Regel würde die erste Line des in Listing \ref{lst: family record example} vorgestellten Family-Records als Eingabe akzeptieren. Sollen nun ebenfalls HUSB- und WIFE Structures als Substructures des Family Records akzeptiert werden, könnte die Nearley Grammatik wie folgt erweitert werden:
\\ \\
\begin{minipage}{1.0\textwidth} \small
	\begin{lstlisting}
		record_FAM_Substructs
			-> "0"  %D  %Xref  %D  "FAM"  %EOL 
			|  record_FAM  record_FAM_Substructs:+
		
		record_FAM_Substructs 
			-> "1"  %D  "HUSB"  %D  %Xref  %EOL
			|  "1"  %D  "WIFE"  %D  %Xref  %EOL 
	\end{lstlisting}
	\captionof{lstlisting}{Nearley Regel zum parsen eines Family Records mit HUSB- und WIFE Substructures}
	\label{lst: nearley regel family record with husb and wife}
\end{minipage}
\\ \\
Auf diese Weise nimmt würde der Nearley Parser einen Family Record ohne Substructures und einen Family Record mit beliebig vielen Substructures (in diesem Fall HUSB- und WIFE Structures) als Eingabe akzeptieren. Sollen nun die weiteren Lines aus Listing \ref{lst: family record example} ebenfalls in die Grammatik aufgenommen werden, müssen Regeln für die Datentypen der Payloads des MARR-Events und der NCHI-Structure definiert werden. Die Anzahl der Kinder wird als  \textit{Integer} Datentyp kodiert, also ein Folge von Dezimalziffern. Nach der Gedcom7 Spezifikation dürfen \textit{Integer} Werte nicht leer sein und führende Nullen sind erlaubt, sollten aber vermieden werden. Eine Regel für den Datentyp \textit{Integer} kann also dargestellt werden als
\\ \\
\begin{minipage}{1.0\textwidth} \small
	\begin{lstlisting}
		digit    ->  [0-9]
		Integer  ->  digit:+
	\end{lstlisting}
	\captionof{lstlisting}{Nearley Regel für den Datentyp \textit{Integer}}
	\label{lst: nearley regel integer}
\end{minipage}
\\ \\ 
Für das MARR-Event, also die Hochzeit der Ehepartner der Familie, ist eine \textit{Date Structure} zum Festhalten des Datums der Hochzeit hinterlegt. Dieses Datum wird mit dem Datentyp \textit{DateValue} kodiert, der im Gegensatz zum \textit{Integer} wesentlich mehr Regeln umfasst. Ein \textit{DateValue} kann auf vier verschiedene Weisen dargestellt sein:
\begin{enumerate}
	\item \textit{date}: Ein mehr oder weniger genau spezifiziertes Datum, z.B. ``AFT JULIAN 13 MAR 1998 BCE''
	\item \textit{datePeriod}: Ein Zeitintervall, dass von einem Startdatum bis zu einem Enddatum angegeben wird, z.B. ``FROM 15 FEB 2001 TO 23 MAR 2001''
	\item \textit{dateRange}: Ein ungenaueres Zeitintervall, bei dem nur Grenzen angegeben werden, z.B. ``BET 15 FEB 2001 AND 23 MAR 2001''
	\item \textit{dateApprox}: Eine Schätzung des Datums (ABT x: genaues Datum unbekannt, aber nahe x), z.B. ``ABT 15 FEB 2001''
\end{enumerate}
Diese Zusammenhänge ergeben die folgenden Nearley Regeln für die Definition des Datentyps \textit{DateValue}:
\\ \\
\begin{minipage}{1.0\textwidth} \small
	\begin{lstlisting}
		DateValue   ->  (date | DatePeriod | dateRange | dateApprox):?
		
		date        ->  (calendar  D):?  
						((day  D):?  month  D):?  
						year  
						(D  epoch):?
		datePeriod  ->  ("FROM"  D  date  D):?  "TO"  D  date
		dateApprox  ->  ("ABT" | "CAL" | "EST")  D  date 
		dateRange   ->  "BET"  D  date  D  "AND"  D  date  
					|   "AFT"  D  date  
					|   "BEF"  D  date 
		
		calendar ->  "GREGORIAN" | "JULIAN" | "FRENCH_R" | "HEBREW"
		day      ->  Integer  
		year 	 ->  Integer
		month    ->  Tag
		epoch    ->  "BCE" | Tag
		
		Tag 	 ->  upperCaseLetter  |  digit  |  underscore 
	\end{lstlisting}
	\captionof{lstlisting}{Nearley Regel für den Datentyp \textit{DateValue}}
	\label{lst: nearley regel date}
\end{minipage}
\\ \\ 
Werden all diese Regeln zusammengefasst lässt sich die folgende Grammatik definieren, die den Family Record aus Listing \ref{lst: family record example} als Eingabe akzeptiert:
\\ \\
\begin{minipage}{1.0\textwidth} \small
	\begin{lstlisting}
		record_FAM_Substructs
			-> "0"  %D  %Xref  %D  "FAM"  %EOL 
			|  record_FAM  record_FAM_Substructs:+
		
		record_FAM_Substructs 
			-> "1"  %D  "HUSB"  %D  %Xref  %EOL
			|  "1"  %D  "WIFE"  %D  %Xref  %EOL 
			|  "1"  %D  "NCHI"  %D  Integer  %EOL 
			|  structure_MARR 
		
		structure_MARR
			-> "1"  %D  "MARR" %EOL
			|  structure_MARR  
		
		structure_DATE
			-> "2" %D  "DATE"  %D  DateValue  %EOL
	\end{lstlisting}
	\captionof{lstlisting}{Nearley Grammatik für den Family Record aus Listing \ref{lst: family record example}}
	\label{lst: vollständige nearley grammatik family record}
\end{minipage}
\\ \\ 
\section{Grammatik Generator}
\label{sec: Implementierung - Grammatik Generator}
% Klassendiagramm mit Ablauf 


\section{Gedcom Struktur}
\label{sec: Implementierung - Gedcom Struktur}
% Klassendiagramm 


\section{Gedcom Parser}
\label{sec: Implementierung - Gedcom Parser}
% Kurze Beschreibung wie alles zusammengefügt wird
% Beispielhafter Ablauf


