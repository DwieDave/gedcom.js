\chapter{Zusammenfassung und Ausblick}
\label{chap: Zusammenfassung und Ausblick}
In dieser Arbeit wurde die JavaScript-Bibliothek \textit{gedcom7.js} für das genealogische Austauschformat FamilySearch GEDCOM Version 7 entwickelt und getestet. Dabei wurden dem Leser alle für das Verständnis benötigten theoretischen Grundlagen an die Hand gegeben und anschließend dem Entwicklungsprozess entsprechend zuerst die Konzeption und danach die konkrete Implementierung präsentiert. 
\\ \\
Die Bibliothek setzt sich aus vier Hauptkomponenten zusammen. Das zentrale Element ist der \textsc{Gedcom Parser}, mit dem Dateien im Format Gedcom7 eingelesen werden und in ein \textsc{Dataset} überführt werden können.
Dabei wurde ein besonderer Fokus auf die spezifikationskonforme Handhabung von Gedcom7-Dateien gelegt. Die Syntaxüberprüfung wurde mit Hilfe der JavaScript-Bibliothek \textit{Nearley} umgesetzt, die auf Grund der vielen nützlichen Features wie z.B. die Möglichkeit Postprozessoren für Regeln anzugeben, eine perfekte Wahl für diese Aufgabe darstellt. Die für die Überprüfung zugrundeliegende Grammatik wird mit Hilfe eines \textit{Grammatik Generators} generiert, der die Gedcom7 Spezifikation in eine Nearley-konforme Syntax überführt. Dies bietet neben der Arbeitserleichterung den großen Vorteil, dass so eine unkomplizierte Erweiterbarkeit der Bibliothek garantiert wird und in zukünftigen Arbeiten Features wie Gedcom Extensions auf einfache Weise implementiert werden können. 


Die Datenstrukturen \textsc{Dataset} und \textsc{Structure} stellen die grundlegenden Datenstrukturen für die Verwaltung von eingelesene Gedcom7 Dateien in der Bibliothek \textit{gedcom7.js} dar. Die Klassen implementieren die wichtigsten Methoden zur Administration von Gedcom7 Dateien und können in zukünftigen Arbeiten einfach erweitert werden, da alle grundlegenden Strukturen bereits implementiert sind. Außerdem wurden Klassen für Records und spezielle Datentypen bereitgestellt. Anhand der Klassen \textsc{Gedcom Date} und \textsc{Family} wurde gezeigt, wie tiefergehende Funktionalitäten und Convinience-Funktionen zur einfachen Handhabung für den Benutzer der Bibliothek implementiert werden können. 
Die Datenstrukturen \textsc{Dataset} und \textsc{Structure} stellen die grundlegenden Datenstrukturen für die Verwaltung von eingelesene Gedcom7 Dateien in der Bibliothek \textit{gedcom7.js} dar. Die Klassen implementieren die grundlegenden Methoden zur Administration von Gedcom7 Dateien und können in zukünftigen Arbeiten einfach erweitert werden, da alle grundlegenden Strukturen bereits implementiert sind. Außerdem wurden Klassen für Records und spezielle Datentypen bereitgestellt. Anhand der Klassen \textsc{Gedcom Date} und \textsc{Family} wurde gezeigt, wie tiefergehende Funktionalitäten und Convenience-Funktionen zur einfachen Handhabung für den Benutzer der Bibliothek implementiert werden können. 


Vergleicht man die Bibliothek \textit{gedcom7.js} mit den verwandten Arbeiten, die in Kapitel \ref{chap: Related Work} vorgestellt wurden, wird deutlich, dass \textit{gedcom7.js} wesentlich ausführlicher und umfangreicher ist. Besonders mit der vollständigen Syntaxüberprüfung und die einfache Erweiterbarkeit hebt die Implementierung von anderen Arbeiten ab. 
Vergleicht man die Bibliothek \textit{gedcom7.js} mit den verwandten Arbeiten, die in Kapitel \ref{chap: Related Work} vorgestellt wurden, wird deutlich, dass \textit{gedcom7.js} in allen Belangen überlegen ist. Besonders die vollständige Syntaxüberprüfung und die einfache Erweiterbarkeit heben die Implementierung von anderen Arbeiten ab. 
\newpage
{
\noindent
Abschließend lässt sich festhalten, dass die Bibliothek gedcom7.js eine robuste und spezifikationskonforme Möglichkeit bietet, Gedcom7 Dateien in JavaScript zu lesen, zu verarbeiten und zu schreiben. Die Syntaxprüfung und die automatisierte Grammatikerstellung durch den \textsc{Grammar Generator} sind besonders nützliche Features, die die Erstellung und Manipulation von Gedcom7-Dateien erleichtern und eine solide Grundlage für die Weiterentwicklung des Projekts darstellen.
}