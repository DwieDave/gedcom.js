\chapter{Zusammenfassung und Ausblick}
\label{chap: Zusammenfassung und Ausblick}

In dieser Abhandlung wird unsere Bibliothek gedcom7.js als Hauptthema präsentiert. Diese Bibliothek enthält 
als Hauptkomponenten einen Parser, Grammatiken, einen Grammatik-Generator und die Structure-Klassen und Methoden. 
Diese können verwendet werden, um Gedcom7-Dateien unter Verwendung von node.js zu 
lesen, zu verarbeiten, zu manipulieren, zu erstellen und zu schreiben.
\\\\
Bei der Ausarbeitung und Implementierung wurde ein besonderer Fokus auf die konforme Handhabung von 
Gedcom7-Dateien gelegt. Durch die Grammatikprüfung beim Einlesen eines Datasets wird sichergestellt, 
dass die Datei den Gedcom7-Spezifikationen entspricht. Die Grammatikprüfung wird auch nach\\Änderungen 
durch die Structure-Klassen-Methoden durchgeführt, um sicherzustellen, dass die Daten nach der 
Manipulation noch den Spezifikationen entsprechen.
\\\\
Das Kernmodul der Grammatikprüfung ist der GrammarGenerator. Mit seinem spezifikationsnahen Interface
kann der GrammarGenerator redundante Strukturen kompakt abbilden und daraus effizient Nearley-Grammatiken
erstellen, aus denen JavaScript-Parser generiert werden. Der GrammarGenerator greift hierbei auf die 
URI-Bezeichner der Gedcom-Spezifikation zurück.

Dank des modularen Aufbaus des GrammarGenerators können auch Features wie Gedcom-Extensions implementiert
werden, bei denen eigene "Tags" definiert werden können. Nach der Erweiterung des GrammarGenerators 
können neue grammatikbasierte Parser erzeugt werden.
\\\\  
Insgesamt bietet die Bibliothek gedcom7.js eine robuste und spezifikationskonforme Möglichkeit, 
Gedcom7-Dateien in JavaScript zu verarbeiten. Die Grammatikprüfung und der GrammarGenerator sind 
besonders nützliche Features, die die Erstellung und Manipulation von Gedcom7-Dateien erleichtern
und für eine Weiterentwicklung des Projekts eine sehr solide Grundlage bietet.