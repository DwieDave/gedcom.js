\chapter{Related Work}
\label{chap: Related Work}
Vor Beginn der theoretischen Ausarbeitung der Gedcom7-Bibliothek haben wir eine umfassende Untersuchung der bestehenden 
Gedcom-Bibliotheken durchgeführt, darunter auch einige der beliebtesten Bibliotheken, die in Python oder Java entwickelt wurden. 
Wir haben uns dabei auf zwei Bibliotheken fokussiert und diese als Orientierung für unsere Arbeit herangezogen.

\section{gedcom7code/js-parser}

Die erste Bibliothek ist der js-parser, welcher von Luther Tychonievich, einem Managing Editor von FamilySearch Gedcom,
entwickelt wurde und als minimaler Parser für Gedcom7-Zeilen dient. Dieser Parser, der in JavaScript geschrieben wurde,
basiert auf einer Regular Expression und liefert die einzelnen Bestandteile der Zeile zurück. Die grundlegende Struktur einer 
Gedcom-Zeile kann damit ermittelt werden.

Die vorliegenden Bestandteile dienen als rudimentäre Basis für die Properties unserer Structure-Klasse, von welcher alle 
\emph{Record}-Klassen erben. Dennoch handelt es sich hierbei lediglich um eine Demonstration eines minimalen Parsers und nicht um eine vollständige Bibliothek. Infolgedessen führten wir weitere Recherchen durch.

\section{python-gedcom}

Die zweite Bibliothek, die uns besonders gefallen hat, ist die python-gedcom-Bibliothek. Hier hat uns vor allem die an die 
Spezifikation angelehnte Klassenhierarchie beeindruckt. Für unsere Zwecke der Gedcom7-Bibliothek bot diese eine ideale Vorlage.
Auffällig ist hierbei, dass jeder Gedcom-Record eine eigene Klasse besitzt, die von einer Elternklasse Structure erbt.

Ein weiterer Aspekt, der uns an dieser Bibliothek gefiel, ist der Ansatz, die einzelnen Felder, die eine Zeile eines \emph{Record}s 
oder \emph{Subrecord}s ausmachen, zu speichern. Dieser Ansatz ermöglicht es uns, Gedcom-Dateien konsistent zu lesen, interpretieren,
manipulieren und schreiben.

Allerdings wurde bei der python-gedcom-Bibliothek unserer Meinung nach kaum Wert auf die Prüfung der genauen Spezifikationsvorgaben
gelegt. Records können beliebig hinzugefügt werden und nicht spezifikationskonforme Records können erstellt werden.

Da uns die Einhaltung der Spezifikation sehr wichtig ist, haben wir uns entschieden, in Gedcom7.js eine sehr detaillierte Grammatik
mittels Nearley.js zu erstellen. Diese Grammatik wird beim Parsen der Gedcom-Datei und nach jeder Operation an einem Dataset 
verwendet, um das Dataset auf Korrektheit zu prüfen und somit auch eine korrekte Gedcom-Datei zu gewährleisten.