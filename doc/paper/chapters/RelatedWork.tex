\chapter{Related Work}
\label{chap: Related Work}
Eine umfassende Untersuchung der bestehenden GEDCOM-Bibliotheken wurde vor Beginn der theoretischen Ausarbeitung der Gedcom7-Bibliothek durchgeführt. 
Dabei wurden auch einige der beliebtesten Bibliotheken, die in Python oder Java entwickelt wurden, untersucht. Zwei Bibliotheken wurden dabei als Orientierung 
für die Arbeit herangezogen.

\section{gedcom7code/js-parser}

Der js-parser, welcher von Luther Tychonievich, einem Managing Editor von FamilySearch Gedcom, entwickelt wurde, dient als minimaler Parser für Gedcom7-Zeilen. 
Dieser Parser, der in JavaScript geschrieben wurde, basiert auf einer Regular Expression und liefert die einzelnen Bestandteile einer Zeile zurück. Die grundlegende 
Struktur einer GEDCOM-Zeile kann damit ermittelt werden.

\section{python-gedcom}

Die zweite Bibliothek, welche interessante Konzepte bezüglich der Handhabung der GEDCOM-Strukturen liefert, ist die \textit{python-gedcom}-Bibliothek. 
Die an die GEDCOM-Spezifikation angelehnte Klassenhierarchie bot für die Zwecke der Gedcom7-Bibliothek eine ideale Vorlage. Jeder Gedcom-Record besitzt eine eigene Klasse, 
die von einer Elternklasse Structure erbt.

Ein weiterer Ansatz dieser Bibliothek ist das Speichern der einzelnen Felder in den Klassen für \textit{Record}s oder \textit{Subrecord}s, die eine GEDCOM-Zeile ausmachen. 
Dies ermöglicht das konsistente Lesen, Interpretieren, Manipulieren und Schreiben von GEDCOM-Dateien

Bei der python-gedcom-Bibliothek wurde die Prüfung der genauen Spezifikationsvorgaben wenig beachtet. Records können somit beliebig
hinzugefügt werden und nicht spezifikationskonforme Records können erstellt und gespeichert werden.
\newpage
Da die Einhaltung der Spezifikation jedoch sehr wichtig ist, wurde in Gedcom7.js eine sehr detaillierte Grammatik mittels Nearley.js erstellt. Diese Grammatik wird
beim Parsen der Gedcom-Datei und nach jeder Operation an einem Dataset verwendet, um das Dataset auf Korrektheit zu prüfen und somit auch eine korrekte Gedcom-Datei
zu gewährleisten.