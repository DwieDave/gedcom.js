\chapter{Konzept}
\label{chap: Konzept}

\begin{figure}[b]
	\centering
	\includegraphics[width=0.8\textwidth]{images/konzept_allgemein.png}
	\caption{Allgemeiner Aufbau}
	\label{fig: Allgemeiner Aufbau}
\end{figure}

Die Bibliothek \textit{gedcom7.js} lässt sich wie in \hyperref[fig: Allgemeiner Aufbau]{Abbildung 4.1} dargestellt in vier logische Teile gliedern. Das zentrale Element ist der \textsc{Gedcom Parser}, mit dem Dateien oder Strings im Format Gedcom7 eingelesen werden und mit Hilfe von \hyperref[sec: Nearley]{\textit{Nearley}} auf Korrektheit der Syntax überprüft werden können. Die dafür zugrundeliegende Grammatik wird mit Hilfe eines \textit{Grammatik Generators} generiert, der die in \cite{GEDCOM} definierte Spezifikation in eine nearley-konforme Syntax überführt. Die so eingelesenen Informationen werden in Gedcom Datenstrukturen gespeichert, die verändert und erweitert werden und anschließend im Format Gedcom7 ausgegeben werden können. In den folgenden Abschnitten werden die vier Teile und das Zusammenspiel dieser in detaillierter Form vorgestellt.

%========================================================================================
% SECTION: GEDCOM GRAMMATIK
%========================================================================================
\section{Gedcom Grammatik}
\label{sec: Konzept - Gedcom Grammatik}
Eine wichtige Anforderung für die Bibliothek ist, dass die Syntax von Dateien oder Strings, die eingelesen werden, gemäß der Gedcom7-Spezifikation überprüfbar sein soll. Da die Gedcom Datenstrukturen veränderbar und erweiterbar sein sollen, ist es wichtig, dass eine Syntaxüberprüfung nach Änderungen auf einfache Weise möglich ist. Umgesetzt wird diese Syntaxüberprüfung mit Hilfe des in \hyperref[sec: Nearley]{Kapitel 2.3} vorgestellten JavaScript-Parser-Toolkits \textit{Nearley}. Mit \textit{Nearley} können auf einfach Weise, menschenlesbare Grammatiken erstellt und zu einem \textit{Nearley-Parser} kompiliert werden. Von großem Vorteil ist dabei, dass Features wie Postprozessoren und die Implementierung eines Lexers unterstützt werden. 

\subsection{Pre- und Postprozessor}
\label{subsec: Konzept - Gedcom Grammatik - Pre- und Postprozessor}
Standardmäßig packt ein \textit{Nearley-Parser} jedes Zeichen, das mit einer Regel übereinstimmt, in ein Array \cite{NearleyDoc}. Bei komplexeren Grammatiken wie der Gedcom7 Spezifikation führt dies dazu, dass sehr viele Arrays innereinander verschachtelt werden, sodass schnell zweistellige Verschatlungsgrade erreicht werden, was ein weiterarbeiten mit den Ergebnissen erschwert. Mit Hilfe von Postprozessoren können jeder \textit{Nearley Regel} Verarbeitungsanweisungen zugewiesen werden, sodass die Ergebnisse beispielsweise im JSON-Format zurückgegeben werden. Auf diese Weise können die eingelesenen Dateien bereits bei der Syntaxüberprüfung in eine passende Darstellungsform gebracht werden, sodass eine leichte Überführung in die passende Gedcom Datenstruktur möglich ist.

Desweiteren kann ein Lexer verwendet werden, um die Arbeit mit \textit{Nearley} zu optimieren. Ein \textit{Nearley-Parser} teilt die Eingabedaten standardmäßig in einen Strom von einzelnen Zeichen, die sequentiell abgearbeitet werden, was auch als \textit{Scannerless Parsing} bezeichnet wird. Ein Lexer ist eine Art Preprozessor, der die Eingabedaten in größere Einheiten, die sog. \textit{Tokens} zusammenfasst \cite{NearleyDoc}. Auf diese Weise wird der Aufwand beim Parsen verringert und die Interpretation der Eingabedaten fällt oft leichter. Ein einfaches Beispiel hierfür ist eine Regel die einen Zahlenwert erwartet. Ist der Eingabewert beispielsweise "137", würde ein \textit{Nearley-Parser} standardmäßig jede Ziffer einzeln einlesen und im Postprozessor müsste definiert werden, dass die aufeinanderfolgenden Ziffern als ein Zahlenwert interpretiert werden sollen. Mit Hilfe eines Lexers könnte eine einfache Regel definiert werden, die den kompletten Zahlenwert als ein Token vorverarbeitet. Im Rahmen dieser Arbeit wurde der JavaScript Lexer \textit{Moo.js} \cite{MooDoc} verwendet. \textit{Moo.js} zeichnet sich durch seine Geschwindigkeit\footnote{Laut den Entwicklern ist Moo.js der schnellste JavaScript-Lexer und $\sim$2-10 mal schneller als herkömmliche Lexer \cite{MooDoc}.} aus und wird von \textit{Nearley} als Lexer unterstützt. 

\subsection{Nearley-Parser für Gedcom7}
\label{subsec: Konzept - Gedcom Grammatik - Nearley-Parser für Gedcom7}
Da die Gedcom7- sowie die Nearley Syntax beide auf EBNF-Sprachkonzepten basieren, lässt sich die Gedcom7 Spezifikation ohne weiteres in eine Nearley Grammatik übersetzten, die dann zu einem Nearley-Parser kompiliert werden kann. Wird diesem Nearley-Parser eine Gedcom7-Datei (.ged) kodiert als UTF-8 Zeichenkette übergeben, erfüllt dieser die folgenden zwei Aufgaben: 


\vspace{1em}
\textbf{1. Überprüfung der Gedcom7 Syntax} \vspace{0.5em} \\
Der Nearley-Parser überprüft den übergebenen Gedcom7-String Line für Line, indem er alle Zeichen (bzw. Tokens) sequentiell liest, bis ein End-Of-Line (EOL) Token  gefunden wird. Nach jedem Zeichen das eingelesen wird, überprüft der Parser, welche in der Grammatik definierten Regeln durch das neu eingelesene Zeichen nicht mehr mit der Zeichenkette übereinstimmen und verwirft diese. Wird ein EOL Token gelesen werden die Postprozessoren aller übereinstimmenden Regeln ausgeführt und ein Array mit den Ergebnissen dieser Postprozessoraufrufe als Ergebnis der Line zurückgegeben. Da die Gedcom7 Grammatik nicht mehrdeutig ist, findet der Parser bei korrekter Gedcom7 Syntax immer ein eindeutiges Ergebnis\footnote{Hier Kapitel ansprechen in dem über ambigious grammar geredet wird, bei level problem} (d.h. beim Erreichen des EOL Tokens ist maximal eine übereinstimmende Regel übrig). Werden bei diesem Prozess alle Regeln der Grammatik ausgeschlossen, bevor ein EOL Token gelesen wird, ist die Syntax des übergebenen Gedcom7-Strings nicht korrekt und ein Syntaxfehler kann erzeugt werden. Da die Zeichenkette sequentiell abgearbeitet wird, kann bei auftretendem Fehler genau aufgezeigt werden, welche Line und welches Zeichen fehlerhaft sind.

\vspace{1em}
\textbf{2. Extrahieren der Strukturinformationen} \vspace{0.5em} \\
Eine weiterer Aufgabe des Nearley-Parsers ist es, die Strukturinformationen des Gedcom7-Strings zu extrahieren, sodass im nächsten Schritt eine einfache Überführung in entsprechende Gedcom Datenstrukturen möglich ist. Durch den sequentiellen Aufbau einer Gedcom7-Datei wird eine Struktur stets vor seinen Substrukturen definiert. Da das erste Token jeder Line stets das Level der Line repräsentiert, kann der Nearley-Parser die Abhängigkeiten der Lines zueinander zuordnen und es ist zu jedem Zeitpunkt eindeutig, welcher Superstruktur eine Struktur zugeordnet werden soll. Folgende Informationen können also durch den Nearley-Parser extrahiert werden: 
\begin{itemize}
	\item \textbf{URI}: Auch wenn bestimtme Structuretypes denselben Tag besitzen, kann aus der Kombination von Level und Tag die eindeutige Gedcom URI bestimmt werden
	\item \textbf{Datentyp}: Sofern ein Payload in der Line vorhanden ist, kann mit Hilfe der URI der Datentyp des Payloads bestimmt werden
	\item \textbf{Superstruktur}: Zu jeder Line kann die entsprechende Superstruktur angegeben werden, sofern es sich nicht um einen Record (Structure mit Level 0) handelt, die keine Superstructure besitzen
	\item \textbf{Substrukturen}: Hat eine Sturktur eine oder mehrere Substrukturen, können diese auf Basis des Levels der Lines bestimmt werden
\end{itemize}

%========================================================================================
% SECTION: GRAMMATIK GENERATOR
%========================================================================================
\section{Grammatik Generator}
\label{sec: Konzept - Grammatik Generator}
% Warum Grammatik Generator -> Anforderung erweiterbar
% Wie umgesetzt? -> JS-Definitionsdateien einlesen 
% Ablauf -> einlesen, grammatik erstellen, kompilieren,
In der Gedcom7 Spezifikation werden 181 Structuretypes verteilt auf 7 Records definiert, die alle in einer Line der Form
\begin{center}
	Level  D  [Xref  D]  Tag  [D  LineVal]  EOL
\end{center}
dargestellt werden. Sollen diese Structuretypes in eine Nearley Grammatik überführt werden, muss für jede dieser Strukturen und jede mögliche Kombination an Substrukturen eine Regel erstellt werden. Da dies eine sehr repetitive Aufgabe ist und sich die Regeln nur an bestimmten Stellen unterscheiden, lässt sich die Grammatikerstellung durch einen Grammatik Generator automatisieren. Dazu können Definitionsdateien erstellt werden, die die für alle Structuretypes die folgenden Informationen bereithalten: 
\begin{itemize}
	\item \textbf{URI}: Die URI des Structuretypes wird benötigt, um eine Struktur eindeutig zuordnen zu können 
	\item \textbf{LineType}: Der LineType gibt an, wie die Line aufgebaut ist (Cross-Reference-Identifier vorhanden? Payload vorhanden?)
	\item \textbf{Datatype}: Sofern ein Payload in der Line vorhanden ist, kann über den Datatype die Syntax des Payloads ermittelt werden
	\item \textbf{Tag}: Der Tag wird benötigt, damit die Nearley Regeln eindeutig sind
	\item \textbf{Substructures}: In der Gedcom7 Spezifikation sind für alle Structuretypes alle möglichen Substructures definiert. Mit dieser Information können alle korrekten Fälle in Nearley Regeln abgebildet werden
	\item \textbf{Level}: Um eine eindeutige Grammatik zu generieren, müssen die Level mit denen ein jeweiliger Structuretype auftreten kann, zwingend mit angegeben werden. Da in der gedcom7 Spezifikation TAGs mehrfach für verschiedene Typen verwendet werden, kann nicht einfach ein generisches Level für die Regeln verwendet werden, das ganzzahlige Werte akzeptiert, da die entstehende Grammatik damit mehrdeutig wäre. Ein Beispiel hierfür sind die Structures \textit{g7:HEAD-DATE} und \textit{g7:DATE-exact} im Gedcom Header. Mit einem generischen Level wären die Regeln für beide Structuretypes identisch mit 
	\begin{center}
		Level  D  "DATE"  D  DateExact  EOL
	\end{center}
	Wird eine solche Line als Substructure eines Header Records von dem Nearley Parser gelesen, kann dieser nicht entscheiden, ob es sich um ein \textit{g7:HEAD-DATE} oder ein \textit{g7:DATE-exact} handelt und würde somit zwei Ergebnisse aufrecht erhalten. Um diese Mehrdeutigkeit zu verhindern, wird das Level in der Definition angegeben.
\end{itemize}
Anhand dieser Informationen kann der Grammatik Generator automatisiert Nearley Regeln formulieren. Diese Regeln können zu einer Grammatik zusammengefasst und anschließend vom Generator zu einem Nearley-Parser kompiliert werden. Anhand des LineTypes kann der Generator den Regeln die passenden Postprozessoren zuweisen, die für das Extrahieren der Strukturinformationen zuständig sind. Auf diese Weise kann ein voll funktionaler Nearley-Parser automatisiert generiert werden, der die Gedcom7-Syntax vollständig parsen und alle für die weitere Verarbeitung benötigten Informationen extrahieren kann. 


Ein weitere großer Vorteil an dieser Automatisierung ist, dass zur Erfüllung der Anforderung der einfachen Erweiterbarkeit der Bibliothek beigetragen wird. Sollte die Bibliothek in zukünftigen Projekten durch neue Strucutretypes o.ä. erweitert werden, ist dies auf einfache und verständliche Weise durch das Hinzufügen neuer Einträge in die Strukturdefinitionen möglich. Desweiteren bildet der Grammatik Generator ein Fundament für einen wichtigen Use-Case, der in weiterführenden Arbeiten adressiert werden sollte: der Möglichkeit Extensions zu definieren. Die Gedcom7 Spezifikation definiert die wichtigsten Strukturen zur Speicherung genealogischer Informationen - für alle Informationen die über diese Standardstrukturen hinausgehen, müssen Extensions definiert werden. Da genealogische Informationen sehr vielfältig sein können, sind Extensions ein probates Mittel, dass in vielen Anwendungen genutzt wird. Mit Hilfe des Grammatik Generators kann die Definition von Extensions umgesetzt werden, indem eine Schnittstelle zum Generator entwickelt wird, die dem Benutzer zur Verfügung gestellt wird. Über diese Schnittstelle kann die Strukturdefinition erweitert werden und anschließend die Grammatik neu generiert und kompiliert werden. Auf diese Weise könnte die Bibliothek auf die Anforderung aller Benutzer angepasst werden. 
%========================================================================================
% SECTION: GEDCOM STRUKTUREN
%========================================================================================
\section{Gedcom Strukturen}
\label{sec: Konzept - Gedcom Strukturen}
\begin{figure}[b]
	\centering
	\includegraphics[width=1\textwidth]{images/konzept_structure.png}
	\caption{Gedcom Strukturen}
	\label{fig: Gedcom Strukturen}
\end{figure}
% Hier auf Gedcom eingehen, welche Records es gibt 
% Vererbungsbaum mit Structures, Records, Dataset usw. 
Die zentrale Struktur in einer Gedcom7 Datei ist das sog. \textit{Dataset}. Jedes Dataset muss mit einem Header Structure beginnen, der Metadaten über das gesamte Dataset beinhaltet und dabei u.a. Aussagen über den Ort und Zeitpunkt der Erstellung und den Ersteller des Datasets selbst machen kann. Die Mindestanforderung an den Header ist, dass die verwendete Gedcom Version in einer dafür vorgesehenen Structure spezifiziert ist. Abgeschlossen wird jedes Dataset mit einer Trailer Line, die das Ende des Datasets repräsentiert. Eine minimales Gedcom7 Dataset sieht also wie folgt aus:
\\ \\
\begin{minipage}{1.0\textwidth} \small
	\begin{lstlisting}
		0 HEAD
		1 GEDC
		2 VERS 7.0
		0 TRLR
	\end{lstlisting}
	\captionof{lstlisting}{Minimales Gedcom7 Dataset}
	\label{lst: minimales dataset}
\end{minipage}
\\ \\
Alle weiteren genealogischen Informationen können in einem oder mehreren Records festgehalten werden. Folgende Records sind in der Gedcom7 Spezifikation definiert:
\begin{itemize}
	\item \textbf{Family (FAM)}: Der Family Record wurde ursprünglich so strukturiert, dass er eine Familie mit einem männlichen Ehemann und einer weiblichen Ehefrau repräsentiert. Um die Migration von bestehenden Gedcom-Dateien auf Gedcom7 zu erleichtern, wurde die Benamung der Strukturen beibehalten. Trotzdem sollen in Gedcom7 Familien, Heirat, Zusammenleben und Adoption unabhängig vom Geschlecht der Partner angegeben werden können und daher das Geschlecht und die Rollen von Partnern nicht aus der Husband- bzw. Wife Struktur abgeleitet werden.
	\item \textbf{Individual (INDI)}: Zusammenstellung von Fakten und Hypothesen über eine Person. Diese können aus verschiedenen Quellen stammen, die durch Quellenangaben dokumentiert werden können.
	\item \textbf{Multimedia (OBJE)}: Eine Referenz zu einer oder mehrerer digitaler Dateien, angereichert mit Informationen über den Inhalt und den Typ der Datei.
	\item \textbf{Repository (REPO)}: Beinhaltet Informationen über Personen oder Institutionen, die eine Sammlung von Quellen besitzen.
	\item \textbf{Shared Note (SNOTE)}: Eine Sammlung von Informationen, die nicht vollständig in andere Strukturen passen. Beispiele wären Forschungsnotizen, alternative Interpretationen oder Argumentationen
	\item \textbf{Source (SOUR)}: Beschreibt eine Quelle, indem auf bestimmte Dokumente oder Verzeichnisse verwiesen wird.
	\item \textbf{Submitter (SUBM)}: Beschreibt eine Person oder eine Institution, die im Dataset enthaltene Informationen begesteuert hat.
\end{itemize}


%========================================================================================
% SECTION: GEDCOM PARSER
%========================================================================================
\section{Gedcom Parser}
\label{sec: Konzept - Gedcom Parser}
% Hier ein Sequenzdiagramm mit dem generellen Ablauf
\begin{figure}[b]
	\centering
	\includegraphics[width=1\textwidth]{images/konzept_sequenz.png}
	\caption{Ablauf Gedcom Parser}
	\label{fig: Sequenz Gedcom Parser}
\end{figure}
